\chapter{Nombre del Capítulo II}

Cada capítulo deberá contener una breve introducción que describe en forma rápida el contenido del
mismo. En este capítulo va el marco teórico. (pueden ser dos capítulos de marco teórico)

\section{Sección 1 del Capítulo II}

Un capítulo puede contener n secciones. La referencia bibliográfica se hace de la siguiente manera:
\cite{Mateos00}

\subsection{Sub Sección}

Una sección puede contener n sub secciones.\cite{Galante01}

\subsubsection{Sub sub sección}

Una sub sección puede contener n sub sub secciones.

\section{Recomendaciones generales de escritura}
Un trabajo de esta naturaleza debe tener en consideración varios aspectos generales:

\begin{itemize}
\item Ir de lo genérico a lo específico. Siempre hay qeu considerar que el lector podría ser alguien no muy familiar con el tema 
y la lectura debe serle atractiva.
\item No poner frases muy largas. Si las frases son de mas de 2 líneas continuas es probable que la lectura sea dificultosa.
\item Las figuras, ecuaciones, tablas deben ser citados y explicados {\bf antes} de que aparezcan en el documento.
\item Encadenar las ideas. Ninguna frase debe estar suelta. Siempre que terminas un párrafo y hay otro a continuación, 
el primero debe dejar abierta la idea que se explicará a continuación. Todo debe tener secuencia.
\end{itemize}


\section{Consideraciones Finales}

Cada capítulo excepto el primero debe contener al finalizarlo una sección de consideraciones que enlacen
el presente capítulo con el siguiente.
